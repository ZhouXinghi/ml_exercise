\documentclass[12pt]{scrartcl}
\usepackage{fullpage,enumitem,amsmath,amssymb,graphicx}

\addtokomafont{section}{\normalsize}
\setlength{\parindent}{0pt}

\newcommand{\vect}[1]{\boldsymbol{#1}}
\newcommand{\ve}{\vect}
\newcommand\R{\mathbb{R}}
\newcommand\E{\mathbb{E}}
\newcommand{\fx}[1]{#1(\vect{x})}
\newcommand{\diff}[1]{\,\mathrm{d}#1}

\DeclareMathOperator*{\argmin}{arg\,min}
\DeclareMathOperator*{\argmax}{arg\,max}
\newcommand{\p}[1][\mathrm{p}]{\mathrm{#1}}

\title{\large Machine Learning Exercise Sheet 1}
\subtitle{\Large Math Refresher}
\author{\large\bfseries Group\_369 \\
        \large Fan \textsc{Xue} -- \texttt{fan98.xue@tum.de} \\
        \large Xing \textsc{Zhou} -- \texttt{xing.zhou@tum.de} \\
        \large Jianzhe \textsc{Liu} -- \texttt{jianzhe.liu@tum.de}}
\date{\large \today}
\begin{document}

  \maketitle
  \vspace{-1cm}
  \noindent\rule{\textwidth}{0.4pt}
  \section*{Problem 7}
  Let $\phi\left( x_1,x_2 \right) = x_1x_2 $, we can observe that for all crosses $\phi\left( x_1, x_2 \right) \le 0$ 
  and for all circles $\phi\left( x_1, x_2 \right)  \ge  0$, which means it is linearly separable.
  We can seperate the crosses and circles with a single hyperplane $\phi\left( x_1, x_2 \right) = 0$.

  \section*{Problem 8}
  On the boundry $\Gamma$, the $\ve{x}$ must realize \[
  p\left( y = 1 \mid \ve{x} \right) = p \left( y = 0 \mid \ve{x} \right) 
  \]
  It is equivalent to \[
  \log \frac{p\left( y = 1 \mid \ve{x} \right) }{p\left( y = 0 \mid  \ve{x} \right) } = 0
  \] 
  We expand \[
  \begin{split}
      \log \frac{p\left( y = 1 \mid \ve{x} \right) }{p\left( y = 0 \mid  \ve{x} \right) } &= 
      \frac{\frac{1}{\left( 2\pi \right) ^{\frac{D}{2}} \left| \Sigma_1 \right|^{\frac{1}{2}}}e^{-\frac{1}{2}\left( \ve{x} - \ve{\mu_1} \right) ^{T}\Sigma_1^{-1}\left( x - \mu_1 \right) } \cdot \pi_1 }{\frac{1}{\left( 2\pi \right) ^{\frac{D}{2}} \left| \Sigma_0 \right|^{\frac{1}{2}}}e^{-\frac{1}{2}\left( \ve{x} - \ve{\mu_0} \right) ^{T}\Sigma_0^{-1}\left( x - \mu_0 \right) } \cdot \pi_0 }\\
      &= \frac{1}{2}\ve{x}^{T}\left( \Sigma_0^{-1} - \Sigma_1^{-1} \right)\ve{x}
      + \ve{x}^{T}\left( \Sigma_1^{-1}\ve{\mu_1} - \ve{\Sigma_0}^{-1}\ve{\mu}_0 \right) \\
      &\ \ \ \ - \frac{1}{2} \ve{\mu}_1^{T}\ve{\Sigma}_1^{-1}\ve{\mu}_1 + \frac{1}{2} \ve{\mu}_0^{T}\ve{\Sigma}_0^{-1}\ve{\mu}_0 
      + \log \frac{\pi_1}{\pi_0} + \frac{1}{2} \log \frac{\left| \ve{\Sigma}_0 \right| }{\left|\ve{\Sigma}_1\right|} \\
      &= \ve{x}^T \ve{A}\ve{x} + \ve{b}^T\ve{x} + c \\
  \end{split}
  \]  
  where we define \[
  \begin{split}
      \ve{A} &= \frac{1}{2}\left( \ve{\Sigma}_0^{-1} -\ve{ \Sigma}_1^{-1} \right) \\
      \ve{b} &=  \Sigma_1^{-1}\ve{\mu_1} - \ve{\Sigma_0}^{-1}\ve{\mu}_0\\
      c &=  - \frac{1}{2} \ve{\mu}_1^{T}\ve{\Sigma}_1^{-1}\ve{\mu}_1 + \frac{1}{2} \ve{\mu}_0^{T}\ve{\Sigma}_0^{-1}\ve{\mu}_0 
          + \log \frac{\pi_1}{\pi_0} + \frac{1}{2} \log \frac{\left| \ve{\Sigma}_0 \right| }{\left|\ve{\Sigma}_1\right|} \\
  \end{split}
  \] 


  

 
\end{document}
